\documentclass[a4paper,table,fontsize=8pt,DIV=6,enabledeprecatedfontcommands]{scrbook}
\recalctypearea

\usepackage[a4paper,twoside]{geometry}

\usepackage{fontspec}
\usepackage{fontawesome}
\usepackage{xcolor}
\usepackage{tikz}
\usepackage{fancyhdr}
\pagestyle{fancy}

\usetikzlibrary{shapes}
\usepackage{circuitikz}

\usepackage{booktabs,tabularx,makecell,ltxtable,caption,lscape,needspace}

%%\let\savedegree\degree
%%\let\degree\relax
\usepackage{mathabx}
%%\let\degree\savedegree

\usepackage{starfont}
\newcommand{\showboth}[1]{\starfontsans #1 & \starfontserif #1}

\renewcommand{\thefootnote}{\fnsymbol{footnote}}

\usepackage{graphicx}

\setlength{\LTleft}{0pt}
\usepackage{multirow}
\usepackage{xfrac}
\usepackage{xfrac}
%%\usepackage{amssymb}

\definecolor{pastelgreen}{rgb}{0.47, 0.87, 0.47}
\definecolor{kellygreen}{rgb}{0.3, 0.73, 0.09}
\definecolor{pastelorange}{rgb}{1.0, 0.7, 0.28}
\definecolor{darkorange}{rgb}{1.0, 0.55, 0.0}
\definecolor{lightpastelpurple}{rgb}{0.69, 0.61, 0.85}
\definecolor{darkpastelpurple}{rgb}{0.59, 0.44, 0.84}
\definecolor{pastelred}{rgb}{1.0, 0.41, 0.38}
\definecolor{darkpastelred}{rgb}{0.76, 0.23, 0.13}
\definecolor{ferrarired}{rgb}{1.0, 0.11, 0.0}
\definecolor{ghostwhite}{rgb}{0.97, 0.97, 1.0}
\definecolor{lightgray}{rgb}{0.83, 0.83, 0.83}

%%\setmainfont[Ligatures=TeX]{TeX Gyre Pagella}
\setsansfont{Fira Sans}
\setmonofont{Inconsolata}

\usepackage[os=mac]{menukeys}
\renewmenumacro{\keys}[+]{shadowedroundedkeys}
\renewmenumacro{\menu}[>]{angularmenus}

\makeatletter
%%\renewcommand{\maketitle}{{\centering\sffamily{\LARGE\bfseries\@title}\par\vskip\baselineskip{\large\@date}\par}\vskip3\baselineskip}
% nifty commands by Paul Gaborit from http://tex.stackexchange.com/a/236891/226
\def\setmenukeyswin{\def\tw@mk@os{win}}
\def\setmenukeysmac{\def\tw@mk@os{mac}}
\makeatother

\newcommand\enhex[1]{%
    \tikz[baseline=(X.base)]
    \node (X) [draw, regular polygon, regular polygon sides=6] {\strut #1};}

\newcommand\enhexsmall[1]{%
    \tikz[baseline=(X.base)]
    \node (X) [draw, regular polygon, regular polygon sides=6] {#1};}

\newcommand\encircle[1]{%
    \tikz[baseline=(X.base)]
    \node (X) [draw, shape=circle, inner sep=0] {\strut \sffamily{#1}};}

\newcommand\pulsegen[1]{%
\begin{circuitikz}[scale=#1,transform shape]
  \draw (0,0) 
        to [C] (1,0); 
\end{circuitikz}
}
    

\title{High Frontier Gazetteer}
\author{Alex Mauer}
\date{2018-10-09}

\begin{document}

\maketitle

\section{Cards}
\subsection{Crew}
\renewcommand{\arraystretch}{1.25}
\LTXtable{\textwidth}{crew}

\subsection{Thrusters}
\LTXtable{\textwidth}{thrusters}

\newgeometry{left=3cm}
\subsection{Robonauts}
\LTXtable{\textwidth}{robonauts}
\restoregeometry

\subsection{Refineries}
\LTXtable{\textwidth}{refineries}

\subsection{Generators}
\LTXtable{\textwidth}{generators}

\subsection{Reactors}
\LTXtable{\textwidth}{reactors}

\subsection{Radiators}
\LTXtable{\textwidth}{radiators}

\section{Sites}

\subsection{ISRU 4}

All sites which can be prospected with an ISRU-4 robonaut
\subsubsection{Type C}
\LTXtable{\textwidth}{4hydrationC}
\needspace{4\baselineskip}
\subsubsection{Type D}
\LTXtable{\textwidth}{4hydrationD}
\subsubsection{Type M}
\LTXtable{\textwidth}{4hydrationM}
\subsubsection{Type S}
\LTXtable{\textwidth}{4hydrationS}
\subsubsection{Type V}
\LTXtable{\textwidth}{4hydrationV}

\subsection{ISRU 3}
All sites which can be prospected with an ISRU-3 robonaut
\subsubsection{Type C}
\LTXtable{\textwidth}{3hydrationC}
\pagebreak
\subsubsection{Type D}
\LTXtable{\textwidth}{3hydrationD}
\subsubsection{Type M}
\LTXtable{\textwidth}{3hydrationM}
\needspace{4\baselineskip}
\subsubsection{Type S}
\LTXtable{\textwidth}{3hydrationS}
\subsubsection{Type V}
\LTXtable{\textwidth}{3hydrationV}

\subsection{ISRU 2}
All sites which can be prospected with an ISRU-2 robonaut
\subsubsection{Type C}
\LTXtable{\textwidth}{2hydrationC}
\subsubsection{Type D}
\LTXtable{\textwidth}{2hydrationD}
\subsubsection{Type M}
Any M site which can be prospected at ISRU-2 may also be prospected at ISRU-3.
\LTXtable{\textwidth}{2hydrationM}
\subsubsection{Type S}
\LTXtable{\textwidth}{2hydrationS}
\subsubsection{Type V}
\LTXtable{\textwidth}{2hydrationV}

\pagebreak
\subsection{ISRU 1}
All sites which can be prospected with an ISRU-1 robonaut
\subsubsection{Type C}
\LTXtable{\textwidth}{1hydrationC}
\pagebreak
\subsubsection{Type D}
Any D site which can be prospected at ISRU-1 may also be prospected at ISRU-2.
\LTXtable{\textwidth}{1hydrationD}
\subsubsection{Type M}
\LTXtable{\textwidth}{1hydrationM}
\needspace{4\baselineskip}
\subsubsection{Type S}
\LTXtable{\textwidth}{1hydrationS}
\needspace{4\baselineskip}
\subsubsection{Type V}
\LTXtable{\textwidth}{1hydrationV}

\subsection{ISRU 0}
All sites which can be prospected with an ISRU-0 (or less) robonaut
\subsubsection{Type C}
\LTXtable{\textwidth}{0hydrationC}
\subsubsection{Type D}
Any D site which can be prospected at ISRU-0 may also be prospected at ISRU-1 or ISRU-2.
\LTXtable{\textwidth}{0hydrationD}
\subsubsection{Type M}
\LTXtable{\textwidth}{0hydrationM}
\needspace{4\baselineskip}
\subsubsection{Type S}
\LTXtable{\textwidth}{0hydrationS}
\subsubsection{Type V}
Any V site which can be prospected at ISRU-0 may also be prospected at ISRU-1.
\LTXtable{\textwidth}{0hydrationV}

\subsection{Special features}
\subsubsection{Atmosphere}
There are eight Atmospheric sites which work with the card \textit{Ionosphere lasing} and may allow an ISRU scoop operation
\LTXtable{\textwidth}{atmospheric}

\end{document}